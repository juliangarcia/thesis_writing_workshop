\documentclass[11pt]{report}
\title{Automatic Programming of Task Specialisation for a Complex Robotic Foraging Task}
\author{Mostafa Rizk}


\begin{document}
\maketitle
\tableofcontents

\chapter{Introduction}
\section{Overview}
The exploration of an area for the purpose of retrieving particular items and returning them to a base is a common task across many application domains. In mining, resources in the ground are retrieved for processing and refinement; in search and rescue, humans in danger are returned to a safe place and in demining, mines are removed from minefields to be disabled. These are all instances of the foraging problem, found in robotics.

\begin{itemize}
	\item Single robots perform foraging in x way but the multi-robot approach has some advantages and swarms have a further advantage (merge with next paragraph) 
\end{itemize}

SR is a branch of robotics wherein a task is achieved using robots that are relatively simple in relation to that task \cite{csahin:IWSS:2004, francesca:Frontiers:2016}. In SR, the desired global behaviour emerges from the local interactions of the robots \cite{csahin:IWSS:2004, brambilla:SI:2013}. For a task like foraging, the intuitive benefit of using a multi-robot approach is that multiple robots are able to cover more area and retrieve more objects in the same span of time. Each robot can take on a different part of the problem space, whether this means taking on a particular geographic area or taking on a role that is different to other robots and thereby partitioning the task into different sub-tasks \cite{ferrante:PLOS_CB:2015}. The benefit of using SR specifically, over a more centralised approach, is that it provides scalability, robustness and flexibility


\section{Motivation}
\section{Research Questions}
\chapter{Progress since confirmation}
\section{Empirical framework}
\section{Challenges}
\section{Results}
\chapter{Future plans}
\section{Something}
\section{Timeline}
\chapter{Thesis outline}

\bibliographystyle{plain}
\bibliography{swarms.bib}

\end{document}
